\index{Base Library@{Base Library}}

The library is implemented in several layers, as described in the previous chapter. The base library is used by all the others, containing functions to deal with the following major functions\+: \begin{DoxyItemize}
\item Safe Strings \item malloc/free\end{DoxyItemize}
These are described in the following sections.\hypertarget{base_library_base_safe_strings}{}\section{Safe Strings}\label{base_library_base_safe_strings}
\index{Safe Strings@{Safe Strings}}

Strings are notorious for causing trouble in C code. Instead of using them, we choose to use a library that implements a safer string. This library is called the Better String Library, and is written by Paul Hsieh. It is available from\+: \href{http://bstring.sourceforge.net}{\tt http\+://bstring.\+sourceforge.\+net}. The basic idea is to have the string and metadata about the string in a {\ttfamily struct}, and the {\ttfamily struct} is used in the code, not the string, as shown in the code fragment below\+: 
\begin{DoxyCode}
\textcolor{keyword}{struct }\hyperlink{structtagbstring}{tagbstring} \{
    \textcolor{keywordtype}{int} mlen;
    \textcolor{keywordtype}{int} slen;
    \textcolor{keywordtype}{unsigned} \textcolor{keywordtype}{char} * data;
\};
\end{DoxyCode}
 Note that the string part of this {\ttfamily struct} is named {\ttfamily data}, while the length of the valid data in the string is stored in {\ttfamily slen}, and the number of bytes allocated to data is stored in {\ttfamily mlen}. As needed, the string part of this struct is allocated, reallocated and deallocated, so that there are fewer errors by the programmer. Note that a {\ttfamily bstring} must be initialized before most of the {\ttfamily bstring} functions can operate on it sensibly. The typical usage of this new datatype is supported by many functions, a few of which are shown below\+:

\begin{DoxyItemize}
\item declaring it\+: {\ttfamily bstring str1} \item declaring and allocating it\+: {\ttfamily bstring str1 = bfromcstr(\char`\"{}\char`\"{})} \item get the c-\/string from the bstring\+: {\ttfamily bdata(bstring b)} \item assign a c-\/string to a bstring\+: {\ttfamily bassigncstr(b, \char`\"{}\char`\"{})} \item sprintf\+: {\ttfamily bassignformat(b, \char`\"{}format\char`\"{}, variables, ...)} \item refer to a character in the string\+: {\ttfamily bchar(b, index)} \item copy a bstring to another bstring, which may not be initialized yet\+: {\ttfamily name = bstrcpy(filename)} \item deallocate\+: {\ttfamily bdestroy(b)} \item strlen\+: {\ttfamily blength(b)} \item strcpy\+: {\ttfamily bassign(a,b)} \item strstr\+: {\ttfamily binstr(b, pos,  c)} \item strcat\+: {\ttfamily bconcat(b, c)} \item constant strings\+: {\ttfamily bsstatic(\char`\"{}blah\char`\"{})} \item reading from a file\+:\end{DoxyItemize}

\begin{DoxyCode}
\hyperlink{structtagbstring}{bstring} theline = bfromcstr(\textcolor{stringliteral}{""});
\textcolor{keyword}{struct }\hyperlink{structbStream}{bStream} * bstream;
FILE *fp;
fp =fopen(\textcolor{stringliteral}{"name"},\textcolor{stringliteral}{"access"});
bstream = bsopen ((bNread) fread, fp);
bsread(theline,bstream,10);
bsclose(bstream);
fclose(fp);
\end{DoxyCode}
\hypertarget{base_library_base_mallocfree}{}\section{Malloc/\+Free}\label{base_library_base_mallocfree}
\index{GMalloc@{GMalloc}} \index{GFree@{GFree}}

We have an implementation of malloc/free called G\+Malloc/\+G\+Free that is used everywhere in our library. This currently is a thin layer on top of the C standard malloc/free. T\+He basic idea here is to not free null pointers, and to have stubs that can be rewritten later if needed, without having to rewrite all of our existing code.

\index{Base Library Macros@{Base Library Macros}} Other things that are available in the {\ttfamily base.\+h} file are\+: definitions for {\ttfamily T\+R\+U\+E} and {\ttfamily F\+A\+L\+S\+E}, macros {\ttfamily M\+I\+N}, {\ttfamily M\+A\+X}, {\ttfamily A\+B\+S}, bstring-\/comparisons\+: {\ttfamily E\+Q\+U\+A\+L}, and {\ttfamily E\+Q\+U\+A\+L\+N}, block-\/copy macros\+: {\ttfamily Byte\+Copy} and {\ttfamily Bstring\+Copy}, and a few other constants and typedefs.

next\+: \hyperlink{hdf_library}{H\+D\+F5 Abstraction Library} 